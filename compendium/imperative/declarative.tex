What you've been practising thus far, is mostly the \key{declarative}
programming paradigm.

\begin{definition}

In a purely declarative programming paradigm, values are \key{immutable}, and
programming is done by describing the relationships between values using
\key{functions}.

\end{definition}

Less formally, in pure declarative programming, we avoid describing the
\key{flow} of the program and resort to pure function application, or rather,
function application is the only connective we've got.  In this paradigm, the
programmer is concerned with the mapping of input values to output values, such
that the mapping defines a \key{class of computations} that the computer can
evoke.  The programmer is less concerned with being particularly clear about
the flow of control or data throughout the program, as this is implicit in the
mapping.

As you may have already noticed, this paradigm can be notoriously difficult to
apply to certain problems, despite (or due to) it's neat mathematical
properties.  What's more, the more accurate student would've already noticed
that you've not done \emph{pure} declarative programming, but rather, already
have experience with a few \key{control flow structures} such as \key{if-} and
\key{case-} \key{statements}\marginExercise{cfsahof}.  Why this is, will become
apparent shortly. We will return to these, and many more control flow
structures, in a bit as well.
