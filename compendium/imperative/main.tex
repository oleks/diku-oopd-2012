\chapter{Imperative programming}

% outline: (declarative -> imperative -> computer -> action -> action sequence -> action
% cluster) | state -> (time | variables -> types -> aggregates -> structures ->
% value vs. reference)
%
% hmm.. how about control flow structures?

% to be introduced in one chapter:
%
% (declarative), imperative, procedural, and structured programming.
%
% under procedural: subroutines vs. coroutines.
%
% primitive types, arrays, structures, value and reference types.
%
% next chapter: data structures.
%
% material for first week: the first two chapters.

% humans can only remember about 4 things at once -- the registers of the
% librarian -- see short-term memory article on wikipedia.

% what's prohibited by the language vs. what's left up to the programmer.
% Consider for instance implicit state references in imperative programming vs.
% functional programming where all input parameters are specified.

\teaser{This chapter introduces some of the most fundamental notions of imperative and
procedural programming.  Many concepts that you already (should) know from your
experience with funcational programming, you'll find reiterated throughout this
chapter.  The intent is to provide a yardstick for distinguishing between
imperative and functional programming.
}

\section{Programming paradigms}

\begin{definition}

A \key{programming paradigm} is a technique for writing a certain \emph{class
of programs} well.

\end{definition}

We will develop the notion of a ``well-written program''
throughout these lecture notes, but initially it suffices to understand
``well'' in the most general sense --- something good, proper, or satisfactory.

A programming language \emph{supports} a programming paradigm, if it
facilitates programming using that paradigm well. The existence and popularity
of a variety of languages and paradigms is considerable empirical evidence that
no paradigm is suitable for all conceivable programs. Indeed, it is not
uncommon to see programming languages that support multiple paradigms, and
well-written programs that utilize multiple paradigms at once.

The absence of good support for a particular paradigm does not imply the
inapplicability of that paradigm in a particular language.  Paradigms are often
composable; but often at considerable cost to the programmers' time and the
well-being of programs. As programmers tend to be lazy, yet aesthetic, this
leads to the evolution of programming languages. For instance, one of the
reasons the original creators of the UNIX operating system created the C
programming language, was that systems programming in assembly, for a complex
system like UNIX, was a rather ``dreary'' task\cite{the-development-of-c}.


% TODO
% A programming language supports a particular paradigm through a
% set of \key{formal rules}. The rules facilitate the use of the paradigm to
% write programs in that language.

% These rules may both facilitate quicker construction of a certain class of
% programs, as well as restrict the programmer in order to guard her from making
% a certain class of logical errors while programming. Hence, a paradigm lets you
% write a certain class of programs more quickly and be more certain in the
% correctness of the programs you write.

% These rules include but are not limited to rules of
% \emph{syntax}, such as how a function should be defined and applied, and rules
% of \emph{semantics}, such as whether a certain function may be applied to
% variables of certain types.

% The syntactic and semantic correctness of the program is often checked
% \key{statically}, i.e. before the program is ever run. In principle, statically
% checked semantics considerably reduce the chances of a certain class of
% programming errors. For instance, you should by now be well-familiar with the
% infamous MoscowML type checker, which didn't let you run a program unless they
% were syntactically well-written and type-checked.




\section{Functional programming}

What you've been practising thus far, is mostly the \key{functional}
programming paradigm.

\begin{definition}

Functional programming treats computation as application of mathematical
functions to \key{immutable} data.

\end{definition}

Programming using the functional paradigm is done by \key{defining}
mathematical functions. The execution of a program is then the application of
some particular \key{entry function} to some, if any, program arguments.
Functions are commonly defined in terms of applications of basic, and
program-defined functions, as well as branching depending on the outputs of
these functions.

A mathematical function defines a relation between a set of possible inputs,
its \key{domain}, and a set of possible outputs, its \key{codomain}. For a
function to be well-defined, every value in its domain of must have a
\key{unique representation}, as must every value in its codomain\footnotemark.
A function can then \key{consume} a representation of a value in its domain,
and \key{produce} a representation of a corresponding value in its codomain.

\footnotetext{Domain and codomain representations do not have to unique
wrt. to one another.}

Due to this pattern of consumption and production of representations, we say
that all data is immutable, i.e. there's no notion of one representation being
mutated into another. Mutation however, may be useful in practise. It is not
unlikely that the output of a function can have a representation very close to
the representation of its input. In such cases, mutation can save us a great
deal of trouble when constructing the output value. For instance, appending a
value to a list, yields the original list with just one more value appended to
it.

Functional programming languages usually hide this aspect from the programmer,
lettting the language itself worry about such ``premature'' matters. Hence,
when programming in a functional programming language, we're less concerned
with the flow of data throughout the program, and more concerned with the
mappings that our functions define.


\section{The von Neumann architecture}

An alternative programming paradigm, the \key{imperative}, is historically
closer to hardware. It emerged as the programming paradigm for computer
architectures stemming from John von Neumann's original proposal for the design
of the digital computer\cite{von-neumann}.  A key aspect of the von Neumann
architecture is that it keeps the computer rather feebleminded, leaving it to
the programmer to be the clever one. Inevitably, no good introduction to the
imperative programming paradigm can commence, without at least a gentle
introduction to the von Neumann architecture.

\begin{definition}

A \key{computer}\footnotemark is an entity with \key{state} and
\key{processors}. A processor performs \key{actions} that change the state of
the computer.

\footnotetext{Although at the beginning of the 20th century, a ``computer'' was
still a profession, today most computers are electronic machines.  We'll use
the words ``computer'' and ``machine'' interchangably.}

\end{definition}

While it may be modern (about time) to speak of computers as machines with
multiple processors, we will restrain ourselves to single processor machines.
Programming for multiple processors raises a range of problematics irrelevant
to the core matter of these lecture notes. At times it will provide for an
interesting discussion to consider how certain aspects map over into
multiprocessor machines, but in such cases they will be mentioned explicitly.

\begin{definition}

``An action is a happening, taking place in a finite period of time and
establishing a well-defined, intended \key{net-effect}.''
\cite{dijkstra-introduction}

\end{definition}

This definition highlights two important points. Firstly, an action takes place
in a finite period of time, say $T_1$. This allows for us to speak of the
points in time $T_n$ and $T_{n+1}$, as the times of action inception and
termination, i.e. $T_{n+1}=T_n+T_1$. Secondly, an action establishes a
well-defined, intended net-effect. This highlights that we are interested in
\emph{deterministic} actions that \emph{change the state} of a computer.

To this end, it makes sense to describe the ``net-effect'' of an action as the
difference between the state of the computer at time $T_n$ and $T_{n+1}$.
However, this notion breaks down as we turn to multiprocessor machines where
the clear benefit of performing multiple actions at once has been utilized.
Therefore, apply this notion with care.

% ``action – a function or a function object that mutates the value of an
% object'' -- Alexander Stepanov, Adobe notes.

As an example of an action, consider eating lunch.


% brewing coffee (perhaps too few instructions)
% going to work (boring instructions)
% elevator (example of coroutines by Knuth)
% doing the dishes
% preparing dinner
% making tea (too few operations)
% following a recipe 
% solving a second degree equation

% As an example of an action, let us consider the taming of a craving for a New
% York Cheesecake\footnote{A lot of potential cheesecakes were harmed in the
% preparation of this example.}, which we'll refer to simply as ``cheesecake''.
% Hence, at point $T_n$, you have a craving for a cheesecake, and at point
% $T_{n+1}$, that craving has already been tamed.

% Rather intuitively, we can come up with a general sequence of sub-actions for
% this action. For instance, we have to \begin{inparaenum}[(1)] \item find a
% recipe for the cheesecake, \item gather ingredients for the cheesecake, and
% \item actually prepare and bake the cheesecake\end{inparaenum}. It is also
% intuitively clear that it is important that these sub-actions occur in this
% particular order, otherwise we'll almost surely end up with a different
% net-effect than originally intended.

% What's less clear, but equally as important is that there are other options to
% taming this craving. For instance, we could go down to the local bakery or
% coffee shop and pay for a ready-made piece of the delicacy. Alternatively, if
% we're billionaires, we might catch a flight to New York and get a piece of the
% ``real'' New York Cheesecake.

% All of these possibilities have their pros and cons. For instance, the home
% made solution is probably the cheapest, and gives you the best options to cater
% the delicacy as you wish, however it requires considerable effort on your side,
% and it takes a considerable amount of time. The local bakery solution is
% probably the quickest, yet probably not the cheapest, and gives you rather few
% catering choices. The last solution is the most expensive, the lengthiest, but
% according to some people may be the most pleasant.

% Depending on the context in which we're referring to the action, it may be
% beneficial to regard it as either an action or this sequence of actions, i.e.
% sub-actions. What's more, the sub-actions themselves can be further broken down
% into their respective sub-actions, recursively. For instance, if we're hungry 

% As an example of an action let us consider solving second degree equations of
% the form $ax^2 + bx + c = 0$.  This is done using the well-known formula:

% $${-b \pm \sqrt{\Delta} \over 2a}.$$

% Where $\Delta$ is commonly known as the discriminant and is defined as follows:

% $$ \Delta = b^2 - 4ac.$$

\newpage

The reason this benefit is ``clear'' comes up when we discuss the notion of
state. We can define the state of a computer as a sequence $S$, of
equally-sized chunks, called \key{words}. The processor is historically only
capable of performing actions on a fixed number of such words at a time.
Usually, this number is \emph{much} smaller than the length of $S$. As a matter
of fact, it is often theoretically beneficial to let $S$ be infinite in size.

For instance, it is not uncommon for a processor to be capable of adding two
numbers together, but it is uncommon for it to be able to perform a summation
of an arbitrary sequence of numbers. If one wishes to perform such a summation,
one must resort to iterated application of the aforementioned binary addition. 

You can imagine the processor in a uniprocessor computer as a lonely librarian
in a long one-dimensional tape library. To comply with our present notion of
state, let the library consist of equally-sized tapes arranged in a linear
fashion. The consequent location of the tapes along the single library
dimension superimposes an \key{address} for every tape relative to some
reference tape. In particular, although the number of tapes can often be left
unspecified, there should always be a discrete finite number of tapes between
any chosen pair of tapes in the library.

The librarian walks atop these tapes, and is so fat that he occupies exactly
the size of one tape. The librarian also has some eye problems, and is
therefore only capable of observing the tape directly below him. Hence, the
librarian is unable to observe multiple randomly distributed tapes
simultaneously. However, like any human, the librarian has a fairly limited
short term memory, but can never-the-less remember a few things at once.
Therefore, he can keep a few tapes in memory while he travels to the other
tapes to observe their values. Once all the values for a particular action are
in his short term memory, the librarian can perform the action and place the
results back in his short term memory.

If the librarian is not going to use the results in the foreseeable future, it
is probably best to write them down somewhere. The only way for the librarian
to do this is to overwrite some library tapes and keep the results there until
they are needed again. This frees up the rather meager short term memory of the
librarian for the performance of other actions.

If we had to hence describe the essential constituents an action, we'd say that
in the time frame $T_1$, the librarian should be able to
\begin{inparaenum}[(1)] \item observe a fixed number of tapes, \item perform an
action with those tapes in his short-term memory, and \item overwrite a fixed
number of tapes with the results\end{inparaenum}.

It is important to reiterate that this sequence of the ``sub-actions'' has to
happen in the fixed time frame of $T_1$, in particular, such that
$T_n+T_1=T_{n+1}$.  This means that regardless of the locations of the tapes in
question, the action must take at most time $T_1$. Yet, the librarian has to
move about in order to observe and overwrite, i.e. to \key{access} various
tapes. In the worst case, the tapes of interest may be at opposite ends of the
(finite) library.  Therefore, the larger the library, the larger the $T_1$ has
to be. 

Instead of letting our concept of time vary on the length of our library, or on
the spatial discontinuity of our actions, we usually take a leap of faith and
let our state $S$ possess a property known as \key{random access}.

\begin{definition}

A sequence with random access, allows for any element of the sequence to
retrieved in equal time, regardless of the sequence size.

\end{definition}

\begin{definition}

A state $S$ is a sequence of equally sized chunks with random access.

\end{definition}

In general, we will refer to any finite-size sequence with consequent element
addressing and random access as an \key{array}. Therefore, any state of a
computer that is finite in size is an array. It is common to refer to the state
of a computer as it's \key{memory} (not to be confused with the short term
memory), and we shall no longer diverge from that jargon.

You can contrast random access to e.g. \key{sequential access}, where elements
further along in the sequence require longer time to retrieve. In a way, this
is akin to giving our librarian teleporting abilities, but really we're just
trusting for an access to a random library tape to be nonessential to the time
frame $T_1$. 

This allows for $T_1$ to be constrained to \begin{inparaenum}[(1)] \item the
number of tapes the librarian needs to perform an action, \item the time it
takes to actually perform the action, and \item the number of tapes that need
to be overwritten as a result of the action\end{inparaenum}. Clearly, much of
this is constrained by the short term memory of the librarian.  Yet, for all
actions, the same restrictive short term memory is used.  Therefore, so long as
the time that it takes to perform an action does not vary considerably from
action to action, the time frame $T_1$ can be regarded as a basic unit of time
for all further time analyses of programs executed on this computer. 

% If we desired a precise estimate of the running time of a program on some
% particular computer, and we had such an abstract analysis at hand, we'd be a
% constant multiplication factor away from the solution. It is practical to
% perform analysis of the running time and space requirements in this manner,
% since they facilitate an analysis of the program regardless of the precise
% details of the underlying architecture.

There's quite a few leaps of faith that we've taken here, and they are worth a
deeper afterthought. First off, we trust for the hardware of the computer to
function properly. For instance, although the librarian has a small short term
memory, we expect him to remember those few tapes par excellence. What's more,
we expect for the tape observations and overwrites to lossless, and so on.

If one is familiar with basic electronics this may seem like quite a leap of
faith to make. However, separation of concerns is an important aspect in the
programmer's profession if she is to remain productive. It has therefore become
conventional for the programmer to assume the correct functioning of a lot of
hardware elements, leaving it to the electrical engineers to make sure that the
assumptions are (almost surely) sound.

On the other hand we've taken the leap of faith that the time frame of an
action does not vary much from action to action, such that it is not completely
absurd to measure the running time of a program by using an action as the basic
unit of time. Interestingly, this does not have to be the case. The intent of
this is to provide means to analyze programs regardless of the actual
architecture on which the program may be run, to a certain extent of course.
Indeed, practice shows that the most popular programs run on a variety of
different computers, varying in the size of memory, speed of the processor,
etc.

Another leap of faith that we've taken is that while some tapes are in short
term memory, the actual tapes in the computer's memory are not overwritten with
new data, i.e. we've assumed that an action is \key{atomic} wrt. the state of
the computer. If we consider a multiprocessor computer, we would have to keep
all other processors from working while some particular processor performs its
action, effectively turning our multiprocessor computer into a uniprocessor
computer. Clearly, this is not very practical, so there exist a series of
techniques that allow for multiple processors to share a single memory on a
given computer, however they are generally outside of the scope of this course.

\marginpar{

``\cyrtext{электронная машина и решения принимала бы быстрее, и просчитывала
намного больше вариантов, и не спешила бы домой в конце рабочего дня}''

Translation: The electronic machine would've made decisions faster, considered
more options, and wouldn't hurry home at the end of the day.

}

It gets to be a pretty tiresome work when the size of $S$ is large, or even
worse, infinite. It is therefore clear that multiple computational units could
do things faster. The real trouble comes along when these computational units
have to use the same tape, but that is outside the scope of this course.

The librarian analogy is pretty lousy, as a librarian that overwrites the
library tapes, is surely to be fired rather promptly. The solution to this
analogy dilemma is to have a range of nonessential tapes that serve as
auxiliaries for the librarian's various operations on essential tapes.

\marginpar{Exercise: XOR swap}

In general, we'd like for our computers, unlike our librarians, to do more than
simply store our data in a neat order. We want for our computers to rearrange,
generate and overwrite data on demand. These, and many more data operations
facilitate our actual goals -- the solution of various real-world problems.
Auxiliary \key{variables}, such as our nonessential tapes, can in this context
be used to store structural information necessary to perform such complex data
manipulations in an efficient manner.

This sequence of steps constitutes and action. Note, that there is no movement
of tapes, just reading and writing, everything else, is done by a combination
of such actions.




% action -> action clusters (Naur)



\newpage

As a by-product, the programmer can often get carried away with some
non-essential details, rather than concentrate on the problem at hand. Indeed,
being ``so close to bear metal'', the paradigm opens up for marvelous
time-spenders on opposing programming frontiers. On the one hand, the
programmer may have to write a great deal of code to even get the simplest
things done, while on the other hand, the programmer can spend a great deal of
time looking for clever tricks to make some part of the program marginally
faster on some particular architecture.

Blah blah conventional CS problem -- abstraction vs. performance.

Time has shown, that neither is a dominating strategy practise. Any
self-respecting programmer knows that premature optimization is the root of all
evil \cite{knuth-goto}, as any self-respecting manager knows not to measure
programmer productivity by the number of lines written.

\marginpar{Exercise: A good computer scientist would quickly come up with a
scheme for generating good for nothing intermediate code, and get paid for not
doing much work at all.}

The more accurate productivity measure seems to be an evaluation of
\begin{inparaenum}[(1)] \item the class of computations that the program can
evoke, and \item the \key{time-} and \key{resource} \key{requirements} of such
evocations\end{inparaenum}. Hence, the programmer's job becomes the more poetic
-- ``designing a class of computations'', rather than the more mundane --
''writing a program''. In essence, it seems more fruitful for us to be more
concerned with the behaviour of programs, rather than their contruction.

This view may be acceptable so long as we're not reminded that programs often
have to evolve, either due to changes in the program requirements, or due to
bugs in the original implementation. What's more, programs are usually evolved
by programmers other than their original programmers. In order to evolve a
program \emph{efficiently}, the epoch programmers have to be able to put
themselves in the shoes of the original programmers, and understand why the
program was constructed the way that it was.  This allows them to evolve the
program efficiently, rather than having to resort to say, rewriting the program
from scratch\footnote{Sometimes, the only way to efficiently evolve a program
is to write it from scratch.}.

\begin{definition}

Efficient program evolution is an evolution that increases program compliance
with its requirements, while introducing the least possible changes to the
original program, and accomodating possible future program  evolution.

\end{definition}

In essence, as programmers, we're playwriters for two rather distinct audiences
(almost) at the same time: the feebleminded computer that has to evoke the
program, and the generally clever, but sometimes feebleminded programmer that
has to evolve our program. In order to contrast these two audiences, we first
need an abstract understanding of the notorious von Neumann architecture.







\begin{quote}"You have perhaps had a dream much like mine: Wouldn't it be nice
to have a glorious system of complete maps of the world, whereby one could (by
turning dials) increase or decrease the scale at will? A similar thing can be
achieved for programs, rather easily, when we give the programs a hierarchic
structure like those constructed step-wise. It is not hard to imagine a
computing system which displays a program in such a way that, by pressing
an-appropriate button, one can replace the name of a routine by its expansion
on the next level, or conversely."\cite{knuth-review-of-sp}\end{quote}

\emph{Fun fact:} (should come after discussion of representation of data types
with multiple words) The rooms at DIKU have a peculiar numbering system that
has a lot to do with the way addressing happens in a computer. If you've ever
seen DIKU from the outside, you might've noticed that, by design, all the
windows of a building are of equal size and the windows are equally spaced,
creating a rather aesthetic outside look. The ``address'' of a room inside DIKU
consists of 3 parts, building, floor and window. Buildings have number codes,
as do floors. This leads you to the correct hall, as on any floor there is only
one hallway. The last part of the address indicates the window at which the
room begins, numbered from east to west, (0-indexed). The analogy breaks down
when we find out that not all rooms in a given hall are of the same size.



\section{State of the art}

Traditionally modules. Literate Programming. C\# has regions.

\input{imperative/algorithms}

\section*{Exercises}

\renewcommand{\theenumi}{\bf\arabic{enumi}}

\begin{enumerate}

\item \exercise{cfsahof} In \referToSection{structures} we discussed how
control flow structures are akin to built-in higher-order functions. Implement
the following control flow structures as polymorphic higher-order functions in
\fun:

\begin{enumerate}

\item A counter loop.

\item A conditional loop.

\item An iterator loop.

\item If-then-else branch.

\item If-then-elseif*-else branch.

\end{enumerate}

\item Implement a HelloWorld program in Go.

\item Write a program that prints itself in Go.

\item Consider the following action cluster:

\begin{codebox}
\Procname{$\proc{XOR}(x:\kw{boolean},y:\kw{boolean})$}
\zi \kw{returns}\ $\left\{
\begin{array}{ll}
\kw{false} & \kw{if}\ (x\ \wedge\ y)\ \vee\ (\neg x\ \wedge\ \neg y)\\
\kw{true} & \kw{otherwise}
\end{array}
\right.$
\end{codebox}

\begin{enumerate}

\item Can a sequence of applications of the $\proc{XOR}$ function be used to
swap two boolean values?

\item Go has a built-in binary infix operator \verb ^ , which corresponds to
running the $\proc{XOR}$ function on the respective bits of the two arguments.
Can you come up with a sequence of Go code that swaps two integers?

\item Can a similar sequence be used to swap two floating point values without
loss of precision?

\end{enumerate}

\item Consider $\proc{Naur-Maximum}\p{A,n}$, discussed in
\referToSection{imperative:algorithms:naur-maximum}. Devise a stable algorithm,
which for a non-empty array $A$, returns a minimum element in $A$.

\end{enumerate}

