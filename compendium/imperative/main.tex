\chapter{Imperative and procedural programming}

This chapter introduces some of the most fundamental notions of imperative and
procedural programming.  Many concepts that you already (should) know you'll
find reiterated throughout this, and future chapters.  These repetitions serve
not to excite you, but are rather means for building up a framework for
understanding various concepts.  In the case of this chapter, it is the
distinction between declarative, imperative and procedural programming.

What you've been practising thus far, is mostly the \key{declarative}
programming paradigm.

\begin{definition}

In a purely declarative programming paradigm, values are \key{immutable}, and
programming is done by describing the relationships between values using
\key{functions}.

\end{definition}

\marginpar{Exercise: Write an if-then-else-if-else-statement function in SML.
(requires explanation of else-if)}

Less formally, in pure declarative programming, we avoid describing the
\key{flow} of the program and resort to pure function application, or rather,
function application is the only flow construct that we've got.  In this
paradigm, the programmer is concerned with the mapping of input values to
output values, such that the mapping provides a class of computations evocable
by the program. The programmer is not concerned with being particularly clear
about the flow of data and control throughout the program, as this is
implicit in the mapping.

\marginpar{A course in Object-oriented programming has historically always
seemed easier on students after a course on Functional Programming.}

As you may have already noticed, this paradigm can be notoriously difficult to
apply to certain problems, despite it's nice mathematical properties. What's
more, the more accurate reader would've already noticed that you haven't done
pure declarative programming, but rather, already have experience with a few
\key{control flow structures} such as \key{if-} and \key{case-}
\key{statements}. Why this is, will become apparent shortly. We will return to
these, and many new control flow structures, in a bit as well.

An alternative programming paradigm, is historically closer to hardware. It
derives from John von Neumann's original proposal for the design of the
general-purpose digital computer\cite{von-neumann}. In this paradigm, the
programmer is much more exposed to the underlying machine architecture.

Computers are historically very mundane. Hence, the programmer often has to be
very explicit about the flow the program in this paradigm. The programmer can
often get carried away with such mundane details, rather than concentrate on
the problem at hand.

So much so, that every once in a while the industry goes back to measuring the
programmer productivity by say, the number of lines of code written. A good
computer scientist would quickly come up with a scheme for generating good for
nothing intermediate code, and get paid for not doing much work at all.

The more accurate measure seems to be an evaluation of the class of
computations evocable by the written program, as well as the time and resource
requirements of such computations. Hence, the programmer's job becomes the more
poetic -- ``designing a class of computations'', rather than the more mundane
-- ''writing a program''.

In essence, it seems more fruitful for the industry to be more concerned with
the net-effect of programs, while the means of getting there are essentially
non-essential. It is important to note, that this is not equivalent to the
declarative approach, since generating non-sensical intermediate function
application is just as easy. In general, we will not be concerned with code
that does nothing, as this is rather pointless.

This is a view acceptable before we realise that software has to be maintained.
What's more, it more often than not has to be maintained by other programmers.

This
was a view acceptable before really big software projects came along -- at
which point it became clear that the code that a programmer writes, has to be
maintained by others.

\begin{definition}

\key{Computers} are entities with \key{state} and \key{computational units}.

\end{definition}

\begin{definition}

Computational units perform \key{actions} that change the state of the computer.

\end{definition}

While it may be modern (about time) to speak of machines as having multiple
computational units, call them \key{cores}, for a brief moment it is best for
our discussion to restrain ourselves to single-core machines.

\begin{definition}

``An action is a happening, taking place in a finite period of time and
establishing a well-defined, intended \key{net-effect}.''
\cite{dijkstra-introduction}

\end{definition}

This definition highlights two important points:

\begin{enumerate}

\item An action occurs over a finite period of time. This allows us to speak of
a point in time $T_n$ and $T_{n+1}$, as the time of action inception and
termination, respectively.

\item An action has a well-defined, intended net-effect. This is to emphasize
our interest in the outcome of an action rather than the action itself.

\end{enumerate}

To this end, it makes sense to describe the ``net-effect'' of an action as the
difference between the state of the machine at time $T_n$ and $T_{n+1}$. This
notion breaks down as we turn to multi-core machines where the clear benefit of
performing multiple actions at once is utilized. Therefore, apply this notion
with care.

The reason this benefit is ``clear'' comes up when we discuss the notion of
state. We can define the state of a computer as a sequence $S$, of
equally-sized chunks, called \key{words}. The computational unit is historically
only capable of performing actions on a fixed number of such words at a time.
Usually, this number is much smaller than $S$. As a matter of fact, it is often
theoretically beneficial to let the size of $S$ be infinite.

You can imagine the computational unit in a single-core computer as a lonely
librarian in a large one-dimensional tape library (state)\footnote{Both exist
in discrete space. Clearly, the librarian is a dot, while the library can be a
line, segment, or ray, all depending on the theoretical applicability of
either.}. By our present notion of state, the library consists of equally-sized
tapes (words) arranged in a linear fashion.  As tapes do not overlap, the
consequent location of the tapes superimposes an \key{address} for every tape.
This is akin to house numbering along one side of a street.


The librarian is always at some particular address in the library, and can only
work with the tape at that location. The librarian is unable to work on
multiple tapes simultaneously, since the librarian moves through space, one
location at a time, to the other tape of interest as a part of that action.
Yet, it is practical for the librarian to be able to work on multiple such
tapes at a time. For instance, the librarian may wish to merge two tapes into
one.

%\marginpar{``электронная машина и решения принимала бы быстрее, и просчитывала
%намного больше вариантов, и не спешила бы домой в конце рабочего дня, и не
%делала бы ничего «по блату»''}


Therefore, if we had to describe what constitutes an action, we'd say that in
some time frame $T_1$, the librarian should be able to \begin{inparaenum}[(1)]
\item grab a few tapes, \item examine the data on these tapes, \item overwrite
them with new data, if necessary, and \item put the tapes back in their
original locations\end{inparaenum}.

It is important to reiterate that this sequence of the sub-actions,
constituting an action, has to happen in the fixed time frame of $T_1$, in
particular, such that $T_n+T_1=T_{n+1}$.  This means that regardless of the
location of the tapes required for some action, the action must take at most
time $T_1$. Yet, the librarian has to move about in order to pick up and put
down various tapes. In the worst case, the tapes of interest may be at opposite
ends of the library. Therefore, the larger the library, the larger the $T_1$
has to be. 

Instead of letting our concept of time vary on the size of our library, we
usually let our state $S$ possess a property known as \key{random access}. In a
way, this is akin to giving our librarian teleporting abilities. This allows
the concept of time to be constrained to \begin{inparaenum}[(1)] \item the
number of tapes a librarian is interested in for any action, \item the time it
takes to examine and overwrite these tapes, as well as \item perform various
computations on the examined tapes\end{inparaenum}. However, we need not dig
deeper down this rabbit hole here.

\begin{definition}

A state $S$ is a sequence of equally sized chunks with random access.

\end{definition}

The librarian analogy is pretty lousy, as a librarian that overwrites the
library tapes, is surely to be fired rather promptly. The solution to this
analogy dilemma is to have a range of empty tapes that serve as auxiliaries for
the librarian's various operations, and to assume that tape overwrites are
lossless. 

\marginpar{Exercise: XOR swap}

In general, we'd like for our computers, unlike our librarians, to do more than
simply store our data in a neat order. We want for our computers to sort, move
about, generate and actually overwrite data on demand. These, and many more
data operations facilitate our actual goal -- the solution of various
real-world problems.

Peter naur.

This is because computer science is both about
managing data and producing it. 

This sequence of steps constitutes and action. Note, that there is no movement
of tapes, just reading and writing, everything else, is done by a combination
of such actions.


It gets to be a pretty tiresome work when the size of
$S$ is large, or even worse, infinite. It is therefore clear that multiple computational
units could do things faster. The real trouble comes along when these
computational units have to use the same tape, but that is outside the scope of
this course.

In particular, multi-core machines usually share the same state.


% action -> action clusters (Naur)


\begin{quote}"You have perhaps had a dream much like mine: Wouldn't it be nice
to have a glorious system of complete maps of the world, whereby one could (by
turning dials) increase or decrease the scale at will? A similar thing can be
achieved for programs, rather easily, when we give the programs a hierarchic
structure like those constructed step-wise. It is not hard to imagine a
computing system which displays a program in such a way that, by pressing
an-appropriate button, one can replace the name of a routine by its expansion
on the next level, or conversely."\cite{knuth-review-of-sp}\end{quote}

\emph{Fun fact:} (should come after discussion of representation of data types
with multiple words) The rooms at DIKU have a peculiar numbering system that
has a lot to do with the way addressing happens in a computer. If you've ever
seen DIKU from the outside, you might've noticed that, by design, all the
windows of a building are of equal size and the windows are equally spaced,
creating a rather aesthetic outside look. The ``address'' of a room inside DIKU
consists of 3 parts, building, floor and window. Buildings have number codes,
as do floors. This leads you to the correct hall, as on any floor there is only
one hallway. The last part of the address indicates the window at which the
room begins, numbered from east to west, (0-indexed). The analogy breaks down
when we find out that not all rooms in a given hall are of the same size.



\section{State of the art}

Traditionally modules. Literate Programming. C\# has regions.

\section*{Exercises}

\renewcommand{\theenumi}{\bf\arabic{enumi}}

\begin{enumerate}

\item \exercise{cfsahof} In \referToSection{structures} we discussed how
control flow structures are akin to built-in higher-order functions. Implement
the following control flow structures as polymorphic higher-order functions in
\fun:

\begin{enumerate}

\item A counter loop.

\item A conditional loop.

\item An iterator loop.

\item If-then-else branch.

\item If-then-elseif*-else branch.

\end{enumerate}

\item Implement a HelloWorld program in Go.

\item Write a program that prints itself in Go.

\item Consider the following action cluster:

\begin{codebox}
\Procname{$\proc{XOR}(x:\kw{boolean},y:\kw{boolean})$}
\zi \kw{returns}\ $\left\{
\begin{array}{ll}
\kw{false} & \kw{if}\ (x\ \wedge\ y)\ \vee\ (\neg x\ \wedge\ \neg y)\\
\kw{true} & \kw{otherwise}
\end{array}
\right.$
\end{codebox}

\begin{enumerate}

\item Can a sequence of applications of the $\proc{XOR}$ function be used to
swap two boolean values?

\item Go has a built-in binary infix operator \verb ^ , which corresponds to
running the $\proc{XOR}$ function on the respective bits of the two arguments.
Can you come up with a sequence of Go code that swaps two integers?

\item Can a similar sequence be used to swap two floating point values without
loss of precision?

\end{enumerate}

\item Consider $\proc{Naur-Maximum}\p{A,n}$, discussed in
\referToSection{imperative:algorithms:naur-maximum}. Devise a stable algorithm,
which for a non-empty array $A$, returns a minimum element in $A$.

\end{enumerate}

