\chapter{Imperative programming}

% outline: (declarative -> imperative -> computer -> action -> action sequence -> action
% cluster) | state -> (time | variables -> types -> aggregates -> structures ->
% value vs. reference)
%
% hmm.. how about control flow structures?

% to be introduced in one chapter:
%
% (declarative), imperative, procedural, and structured programming.
%
% under procedural: subroutines vs. coroutines.
%
% primitive types, arrays, structures, value and reference types.
%
% next chapter: data structures.
%
% material for first week: the first two chapters.

% humans can only remember about 4 things at once -- the registers of the
% librarian -- see short-term memory article on wikipedia.

% what's prohibited by the language vs. what's left up to the programmer.
% Consider for instance implicit state references in imperative programming vs.
% functional programming where all input parameters are specified.

\teaser{This chapter introduces some of the most fundamental notions of imperative and
procedural programming.  Many concepts that you already (should) know from your
experience with funcational programming, you'll find reiterated throughout this
chapter.  The intent is to provide a yardstick for distinguishing between
imperative and functional programming.
}

\section{Programming paradigms}

\begin{definition}

A \key{programming paradigm} is a technique for writing a certain \emph{class
of programs} well.

\end{definition}

We will develop the notion of a ``well-written program''
throughout these lecture notes, but initially it suffices to understand
``well'' in the most general sense --- something good, proper, or satisfactory.

A programming language \emph{supports} a programming paradigm, if it
facilitates programming using that paradigm well. The existence and popularity
of a variety of languages and paradigms is considerable empirical evidence that
no paradigm is suitable for all conceivable programs. Indeed, it is not
uncommon to see programming languages that support multiple paradigms, and
well-written programs that utilize multiple paradigms at once.

The absence of good support for a particular paradigm does not imply the
inapplicability of that paradigm in a particular language.  Paradigms are often
composable; but often at considerable cost to the programmers' time and the
well-being of programs. As programmers tend to be lazy, yet aesthetic, this
leads to the evolution of programming languages. For instance, one of the
reasons the original creators of the UNIX operating system created the C
programming language, was that systems programming in assembly, for a complex
system like UNIX, was a rather ``dreary'' task\cite{the-development-of-c}.


% TODO
% A programming language supports a particular paradigm through a
% set of \key{formal rules}. The rules facilitate the use of the paradigm to
% write programs in that language.

% These rules may both facilitate quicker construction of a certain class of
% programs, as well as restrict the programmer in order to guard her from making
% a certain class of logical errors while programming. Hence, a paradigm lets you
% write a certain class of programs more quickly and be more certain in the
% correctness of the programs you write.

% These rules include but are not limited to rules of
% \emph{syntax}, such as how a function should be defined and applied, and rules
% of \emph{semantics}, such as whether a certain function may be applied to
% variables of certain types.

% The syntactic and semantic correctness of the program is often checked
% \key{statically}, i.e. before the program is ever run. In principle, statically
% checked semantics considerably reduce the chances of a certain class of
% programming errors. For instance, you should by now be well-familiar with the
% infamous MoscowML type checker, which didn't let you run a program unless they
% were syntactically well-written and type-checked.




\section{Functional programming}

What you've been practising thus far, is mostly the \key{functional}
programming paradigm.

\begin{definition}

Functional programming treats computation as application of mathematical
functions to \key{immutable} data.

\end{definition}

Programming using the functional paradigm is done by \key{defining}
mathematical functions. The execution of a program is then the application of
some particular \key{entry function} to some, if any, program arguments.
Functions are commonly defined in terms of applications of basic, and
program-defined functions, as well as branching depending on the outputs of
these functions.

A mathematical function defines a relation between a set of possible inputs,
its \key{domain}, and a set of possible outputs, its \key{codomain}. For a
function to be well-defined, every value in its domain of must have a
\key{unique representation}, as must every value in its codomain\footnotemark.
A function can then \key{consume} a representation of a value in its domain,
and \key{produce} a representation of a corresponding value in its codomain.

\footnotetext{Domain and codomain representations do not have to unique
wrt. to one another.}

Due to this pattern of consumption and production of representations, we say
that all data is immutable, i.e. there's no notion of one representation being
mutated into another. Mutation however, may be useful in practise. It is not
unlikely that the output of a function can have a representation very close to
the representation of its input. In such cases, mutation can save us a great
deal of trouble when constructing the output value. For instance, appending a
value to a list, yields the original list with just one more value appended to
it.

Functional programming languages usually hide this aspect from the programmer,
lettting the language itself worry about such ``premature'' matters. Hence,
when programming in a functional programming language, we're less concerned
with the flow of data throughout the program, and more concerned with the
mappings that our functions define.


\section{The von Neumann architecture}

An alternative programming paradigm, the \key{imperative}, is historically
closer to hardware. It emerged as the programming paradigm for computer
architectures stemming from John von Neumann's original proposal for the design
of the digital computer\cite{von-neumann}.  A key aspect of the von Neumann
architecture is that it keeps the computer rather feebleminded, leaving it to
the programmer to be the clever one. Inevitably, no good introduction to the
imperative programming paradigm can commence, without at least a gentle
introduction to the von Neumann architecture.

\begin{definition}

A \key{computer}\footnotemark is an entity with \key{state} and
\key{processors}. A processor performs \key{actions} that change the state of
the computer.

\footnotetext{Although at the beginning of the 20th century, a ``computer'' was
still a profession, today most computers are electronic machines.  We'll use
the words ``computer'' and ``machine'' interchangably.}

\end{definition}

While it may be modern (about time) to speak of computers as machines with
multiple processors, we will restrain ourselves to single processor machines.
Programming for multiple processors raises a range of problematics irrelevant
to the core matter of these lecture notes. At times it will provide for an
interesting discussion to consider how certain aspects map over into
multiprocessor machines, but in such cases they will be mentioned explicitly.

\begin{definition}

``An action is a happening, taking place in a finite period of time and
establishing a well-defined, intended \key{net-effect}.''
\cite{dijkstra-introduction}

\end{definition}

This definition highlights two important points. Firstly, an action takes place
in a finite period of time, say $T_1$. This allows for us to speak of the
points in time $T_n$ and $T_{n+1}$, as the times of action inception and
termination, i.e. $T_{n+1}=T_n+T_1$. Secondly, an action establishes a
well-defined, intended net-effect. This highlights that we are interested in
\emph{deterministic} actions that \emph{change the state} of a computer.

To this end, it makes sense to describe the ``net-effect'' of an action as the
difference between the state of the computer at time $T_n$ and $T_{n+1}$.
However, this notion breaks down as we turn to multiprocessor machines where
the clear benefit of performing multiple actions at once has been utilized.
Therefore, apply this notion with care.

% ``action – a function or a function object that mutates the value of an
% object'' -- Alexander Stepanov, Adobe notes.

As an example of an action, consider eating lunch.


% \input{imperative/algorithms}

% \section*{Exercises}

\renewcommand{\theenumi}{\bf\arabic{enumi}}

\begin{enumerate}

\item \exercise{cfsahof} In \referToSection{structures} we discussed how
control flow structures are akin to built-in higher-order functions. Implement
the following control flow structures as polymorphic higher-order functions in
\fun:

\begin{enumerate}

\item A counter loop.

\item A conditional loop.

\item An iterator loop.

\item If-then-else branch.

\item If-then-elseif*-else branch.

\end{enumerate}

\item Implement a HelloWorld program in Go.

\item Write a program that prints itself in Go.

\item Consider the following action cluster:

\begin{codebox}
\Procname{$\proc{XOR}(x:\kw{boolean},y:\kw{boolean})$}
\zi \kw{returns}\ $\left\{
\begin{array}{ll}
\kw{false} & \kw{if}\ (x\ \wedge\ y)\ \vee\ (\neg x\ \wedge\ \neg y)\\
\kw{true} & \kw{otherwise}
\end{array}
\right.$
\end{codebox}

\begin{enumerate}

\item Can a sequence of applications of the $\proc{XOR}$ function be used to
swap two boolean values?

\item Go has a built-in binary infix operator \verb ^ , which corresponds to
running the $\proc{XOR}$ function on the respective bits of the two arguments.
Can you come up with a sequence of Go code that swaps two integers?

\item Can a similar sequence be used to swap two floating point values without
loss of precision?

\end{enumerate}

\item Consider $\proc{Naur-Maximum}\p{A,n}$, discussed in
\referToSection{imperative:algorithms:naur-maximum}. Devise a stable algorithm,
which for a non-empty array $A$, returns a minimum element in $A$.

\end{enumerate}

