\begin{definition}

``An action is a happening, taking place in a finite period of time and
establishing a well-defined, intended \key{net-effect}.''
\cite{dijkstra-introduction}

\end{definition}

This definition highlights two important points. Firstly, an action takes place
in a finite period of time, say $T_1$. This allows for us to speak of a point
in time $T_n$ and $T_{n+1}$, respectively as the time of action inception and
termination. In particular, such that $T_{n+1}=T_n+T_1$. Secondly, an action
establishes a well-defined, intended net-effect. This highlights that we are
interested in deterministic actions that change the state of the computer.

To this end, it makes sense to describe the ``net-effect'' of an action as the
difference between the state of the computer at time $T_n$ and $T_{n+1}$.
However, this notion breaks down as we turn to multiprocessor machines where
the clear benefit of performing multiple actions on the same computer at once
has been utilized.  Therefore, apply this notion with care. We will discuss the
notion of ``state'' in depth in a latter section.

% ``action – a function or a function object that mutates the value of an
% object'' -- Alexander Stepanov, Adobe notes.
