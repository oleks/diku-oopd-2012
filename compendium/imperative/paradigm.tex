A \key{programming paradigm} is a technique for writing a certain \emph{class
of programs} well. We will develop the notion of a ``well-written program''
throughout these lecture notes, but initially it suffices to understand
``well'' in the most general sense --- something good, proper, or satisfactory.

A programming language \emph{supports} a programming paradigm if it facilitates
programming using that paradigm well. The existence and popularity of a variety
of languages and paradigms is considerable empirical evidence that no paradigm
is suitable for all conceivable programs. Indeed, it is not uncommon to see
programming languages that support multiple paradigms, and well-written
programs that utilize multiple paradigms at once.

The absence of good support for a particular paradigm in a particular language
does not imply the inapplicability of that paradigm.  Paradigms are often
composable, albeit at considerable cost to the programmers' time and the
well-being of the program. As programmers tend to be lazy yet aesthetic, this
leads to the development of new programming languages. For instance, one of the
reasons the original creators of the UNIX operating system created the C
programming language, was that systems programming in assembler, for a complex
system like UNIX, was a rather ``dreary'' task\cite{the-development-of-c}.


% TODO
% A programming language supports a particular paradigm through a
% set of \key{formal rules}. The rules facilitate the use of the paradigm to
% write programs in that language.

% These rules may both facilitate quicker construction of a certain class of
% programs, as well as restrict the programmer in order to guard her from making
% a certain class of logical errors while programming. Hence, a paradigm lets you
% write a certain class of programs more quickly and be more certain in the
% correctness of the programs you write.

% These rules include but are not limited to rules of
% \emph{syntax}, such as how a function should be defined and applied, and rules
% of \emph{semantics}, such as whether a certain function may be applied to
% variables of certain types.

% The syntactic and semantic correctness of the program is often checked
% \key{statically}, i.e. before the program is ever run. In principle, statically
% checked semantics considerably reduce the chances of a certain class of
% programming errors. For instance, you should by now be well-familiar with the
% infamous MoscowML type checker, which didn't let you run a program unless they
% were syntactically well-written and type-checked.


