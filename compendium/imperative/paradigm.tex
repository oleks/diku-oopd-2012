\begin{definition}

A \key{programming paradigm} is a technique for writing a certain class of
programs well.

\end{definition}

We will develop the notion of a ``well-written program'' throughout these
lecture notes, but initially it suffices to understand ``well'' in the most
general sense: something good, proper, or satisfactory.

A programming language \emph{supports} a programming paradigm, if it
facilitates programming using that paradigm well. The existence and popularity
of a variety of languages and paradigms is considerable empirical evidence that
no language nor paradigm is suitable for all conceivable programs. It is not
uncommon to see programming languages that support multiple paradigms, and
well-written programs that utilize multiple paradigms at once.

The absence of good support for a particular paradigm does not imply the
inapplicability of that paradigm in a particular language.  Paradigms are often
composable; but often at considerable cost to programmer time, and the
well-being of programs. As programmers tend to be lazy, yet aesthetic, this
leads to the evolution of programming languages. For instance, one of the
reasons the original creators of the Unix operating system created the C
programming language, was that systems programming in assembly, for a complex
system like Unix, was a rather ``dreary'' task\cite{the-development-of-c}.

