A \key{programming paradigm} is a technique for writing a certain class of
programs well. A programming language \emph{supports} a particular paradigm if
it provides good facilities for writing programs using that paradigm.

The existence and popularity of a variety of paradigms seems like solid
empirical evidence that not all programs can be written \emph{well} in any
particular paradigm, or at least that the \emph{ultimate} paradigm is yet to be
found.  Indeed, it is not uncommon to see programming languages that support
multiple paradigms, and well-written programs that utilize multiple paradigms
for their various constituents.

At the same time, just because a programming language does not provide
\emph{good} facilities for a particular paradigm, it does not mean that it is
\emph{impossible} to use the paradigm to write programs in that language. It
merely means that using that paradigm in the language might come at a
considerable cost of the programmer's time.

In effect, new programming languages often arise when programmers realise that
a considerable effort is required to write certain classes of programs well
using existing tools. For instance, one of the reasons the original creators of
the UNIX operating system created the C programming language, was that systems
programming in assembler, for such a complex system as UNIX, was a rather
dreary task\cite{the-development-of-c}.

A paradigm is supported by a programming language through \emph{syntactical
constructs}, such as function declarations, and \emph{semantical rules} that
can be checked \key{statically}, i.e.  before the program is ever run. In
principle, this considerably reduces the chances of programming errors. For
instance, you should by now be well-familiar with the infamous MoscowML type
checker, which didn't let you run a program unless the types literally made
sense.

