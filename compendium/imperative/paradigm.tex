A \key{programming paradigm} is a way of writing a certain class of programs
well. A programming language that supports a particular paradigm, simply
provides good facilitates for programming using that paradigm.

Not all programs can be well-written in any particular paradigm. It is not
uncommon to see programming languages that provide descent support for a
variety of different paradigms. At the same time, just because a programming
language does not provide good facilities for a particular paradigm, it does
not mean that this paradigm cannot be forced on top of this language with due
effort.

In effect, new programming languages and paradigms often arise whenever
programmers realise that they have to make a considerable effort to write
certain classes of programs well.  For instance, one of the reasons the
original creators of the UNIX operating system created the C programming
language, was that systems programming in assembler, for such a complex system
as UNIX, was a rather dreary task\cite{the-development-of-c}.

