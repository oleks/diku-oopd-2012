An alternative paradigm, the \key{imperative}, lets you be explicit about such
matters. The paradigm emerged as the prime programming paradigm for computers
with architectures stemming from John von Neumann's original proposal for the
design of the digital computer\cite{von-neumann}. Today, the overwhelming
majority of computers derive from this architecture; and no good introduction
to the imperative paradigm can commence without at least a gentle introduction
to the von Neumann architecture.

A key aspect of the von Neumann architecture is that it keeps the computer
rather feebleminded, leaving it to the programmer to be the clever one. As we
shall discover, we've managed to come up with clever methods of abstraction
that guard us against certain pitfalls, but programming is still an immensely
challenging endeavour; in part, due to the persistent fundamental difficulty of
solving certain kinds of problems automatically.

\begin{definition}

A \key{computer}\footnotemark is an entity with \key{state} and
\key{processors}. A processor performs \key{actions} that modify the state of
the computer.

\footnotetext{Although at the beginning of the 20th century, a ``computer'' was
still a profession, today most computers are electronic machines.  We'll use
the words ``computer'' and ``machine'' interchangably.}

\end{definition}

While it may be modern (about time) to speak of computers as machines with
multiple processors, we will restrain ourselves to single processor machines.
Programming for multiple processors raises a range of problematics irrelevant
to the core matter of these lecture notes. At times, it will provide for an
interesting discussion to consider how certain aspects map over to
multiprocessor machines, but in such cases they will be mentioned explicitly.

\begin{definition}

``An action is a happening, taking place in a finite period of time and
establishing a well-defined, intended \key{net effect}.''
\cite{dijkstra-introduction}

\end{definition}

This definition highlights two important points. Firstly, an action takes place
in a finite period of time, say $T_1$. This allows us to speak of the points in
time $T_n$ and $T_{n+1}$, as the times of action inception and termination,
i.e. $T_{n+1}=T_n+T_1$. Secondly, an action establishes a well-defined,
intended net effect. This highlights that we are interested in actions that do
something, indeed something expected.

To this end, it makes sense to describe the ``net effect'' of an action as the
difference between the state of the computer at time $T_n$ and $T_{n+1}$.
However, this notion breaks down as we turn to multiprocessor machines, where
the clear benefit of performing multiple actions at once has been utilized.
Apply this notion with care.

% ``action – a function or a function object that mutates the value of an
% object'' -- Alexander Stepanov, Adobe notes.

As an example of an action, consider doing the laundry. Given a pile of dirty
laundry, the net effect of this action is the same pile of laundry; but clean.
Let us refer to this action as \function{Do-The-Laundry}.

Clearly, the functional paradigm fits abnormally to this type of problem.
Knitting a pile of laundry --- clean, but otherwise equivalent --- is not
nearly as practical as cleaning the dirty pile. Why should we dispose of a pile
of perfectly good laundry, just because it has gotten a little dirty? The
process of cleaning --- rather than knitting --- is more explicit in the
imperative paradigm.

\function{Do-The-Laundry} has a rather abstract definition at the moment. We've
defined it's net effect, but not how to achieve it. In some contexts, such as
planning chores around the house, this definition may be sufficient. In others,
such as actually doing the laundry, it is insufficient.

As a matter of fact, even in the context of planning chores around the house,
the current definition is insufficient. We have important programs to write ---
chores around the house should not take up an indefinite amount of time. We
have a budget to keep --- doing the laundry should not cost an indefinite
amount of money. These, and other important attributes of the action cannot be
defined without defining how we do the laundry first, indeed because there are
different ways to do the laundry, each with different costs implied.

A processor is a discrete physical entity. As such, it only has a finite set of
actions that it can perform. While possible, it is clearly impractical to
extend the physical processor whenever we needed to do a new kind of action.
This realization is what leads to another, and probably most important marvel
of the von Neumann architecture, a \key{stored-program computer}.

Instead, we seek to build \key{general-purpose} processors, and sequence basic
actions to achieve more complex overall net effects. Given a complex action, we
gradually decompose it into sequences of subactions that overall achieve the
desired net effect. Let us attempt to do this for the \function{Do-The-Laundry}
action:

\begin{codebox}
\Procname{\function{Do-The-Laundry}(a pile of dirty laundry)}
\li split the pile into colors and whites\label{code:do-the-laundry:1}
\li wash colors
\li wash whites
\li make a neat stack of the clean laundry
\li \Return the neat stack
\end{codebox}

This is an example of a \key{routine}, \key{procedure}, \key{method},
\key{process}, \key{function}, \ldots

\begin{flushright}

``What's in a name? that which we call a rose\\
By any other name would smell as sweet;''\\
--- William Shakespeare, Romeo and Juliet.

\end{flushright}

Indeed, a named sequence of actions, that establishes some overall net effect
has many names. We'll adopt the term ``routine'', since all of us have a
routine for doing the laundry and doing the laundry is a rather routine task!

The subactions specified above aren't particularly ``basic''. Indeed, we've
specified the complex \function{Do-The-Laundry} in terms of a series of other
complex actions. These subactions are not well-defined, in the sense that we've
simply used natural language to describe them. Let us consider the sequence of
subactions that would constitute the action on line
\ref{code:do-the-laundry:1}:

\begin{codebox}
\Procname{\function{Split-Whites-And-Colors}(a pile of laundry)}
\li let $whites$ be an empty pile
\li let $colors$ be an empty pile
\li \For each piece of laundry \Do
\li \If peace is white \Then
\li put piece in $whites$
\li \Else
\li put piece in $colors$
\End
\End
\li \Return $whites$ and $colors$
\end{codebox}

