\section{The von Neumann architecture}

An alternative programming paradigm, the \key{imperative}, is historically
closer to hardware. It emerged as the programming paradigm for computer
architectures stemming from John von Neumann's original proposal for the design
of the digital computer\cite{von-neumann}.  A key aspect of the von Neumann
architecture is that it keeps the computer rather feebleminded, leaving it to
the programmer to be the clever one. Inevitably, no good introduction to the
imperative programming paradigm can commence, without at least a gentle
introduction to the von Neumann architecture.

\begin{definition}

A \key{computer}\footnotemark is an entity with \key{state} and
\key{processors}. A processor performs \key{actions} that change the state of
the computer.

\footnotetext{Although at the beginning of the 20th century, a ``computer'' was
still a profession, today most computers are electronic machines.  We'll use
the words ``computer'' and ``machine'' interchangably.}

\end{definition}

While it may be modern (about time) to speak of computers as machines with
multiple processors, we will restrain ourselves to single processor machines.
Programming for multiple processors raises a range of problematics irrelevant
to the core matter of these lecture notes. At times it will provide for an
interesting discussion to consider how certain aspects map over into
multiprocessor machines, but in such cases they will be mentioned explicitly.

\begin{definition}

``An action is a happening, taking place in a finite period of time and
establishing a well-defined, intended \key{net-effect}.''
\cite{dijkstra-introduction}

\end{definition}

This definition highlights two important points. Firstly, an action takes place
in a finite period of time, say $T_1$. This allows for us to speak of the
points in time $T_n$ and $T_{n+1}$, as the times of action inception and
termination, i.e. $T_{n+1}=T_n+T_1$. Secondly, an action establishes a
well-defined, intended net-effect. This highlights that we are interested in
actions that do something, in particular, something expected.

To this end, it makes sense to describe the ``net-effect'' of an action as the
difference between the state of the computer at time $T_n$ and $T_{n+1}$.
However, apply this notion with care. It breaks down as we turn to
multiprocessor machines, where the clear benefit of performing multiple actions
at once has been utilized.

% ``action – a function or a function object that mutates the value of an
% object'' -- Alexander Stepanov, Adobe notes.

As an example of an action, consider doing the laundry. Given a pile of dirty
laundry, the net-effect of this action is that the pile is cleaned. (Naur,
function vs. process).

\begin{codebox}
\Procname{$\proc{Do-The-Laundry}\p{dirtyPile}$}
\li \Return $cleanPile$
\end{codebox}

We can dissect this definition into
\key{pre-conditions} and \key{post-conditions}, and state their relationship in
terms of an \key{implication}, denoted by the symbol $\Rightarrow$:

\begin{equation}
\text{Dirty pile}\Rightarrow \text{Clean pile}.
\end{equation}



More formally, if at time $T_n$ we
have a pile of dirty laundry, and perform an action in the subsequent time slot
$T_1$, then at time $T_{n+1}$ we have a pile of clean laundry.


To define the
net-effect of an action, we state the conditions that must hold at time
$T_{n+1}$, given that certain conditions hold at time $T_n$. We call these
conditions the \key{post-} and \key{pre-conditions}, respectively. Given an
action $a$, we say that if the preconditions hold, then after the action, the
post-conditions must hold, denoted

\begin{equation}
\text{Precondition} \Rightarrow^a \text{Postcondition}.
\end{equation}


The effect
of the action is that a precondition \emph{implies} a post condition, we denote
this with the symbol $\Rightarrow$.

It does not make sense to do the laundry without a pile of dirty laundry, and a
good ``doing the laundry'' action leaves all the laundry clean after the action
took place. We therefore state the following:

 Before the action
commences, you have a pile of dirty laundry; after the action, the laundry is
clean. We call these the \key{pre-} and \key{postconditions} of an action,
respectively. In particular, it does not make sense to do the laundry unless
you have a pile of dirty laundry, and the post-condition that the laundry is
clean is probably useful for future actions.

The post-conditions are not complete. There are certain things that we would
like to do to facilitate further actions. For instance, it is useful if in the
process of doing the laundry we didn't mix clean and dirty laundry, and at best
-- neatly stacked the clean laundry in the closet.


The post-conditions are not
completely well-formed however. In particular, 

where whites are mixed with
colors. Anyone who's ever done their laundry knows that it's no good to mix 

After the action, the clothes should be neatly stacked in the closet. 

% doing the laundry 
