\section{The von Neumann architecture}

An alternative programming paradigm, the \key{imperative}, is historically
closer to hardware. It emerged as the programming paradigm for computer
architectures stemming from John von Neumann's original proposal for the design
of the digital computer\cite{von-neumann}.  A key aspect of the von Neumann
architecture is that it keeps the computer rather feebleminded, leaving it to
the programmer to be the clever one. Inevitably, no good introduction to the
imperative programming paradigm can commence, without at least a gentle
introduction to the von Neumann architecture.

\begin{definition}

A \key{computer}\footnotemark is an entity with \key{state} and
\key{processors}. A processor performs \key{actions} that change the state of
the computer.

\footnotetext{Although at the beginning of the 20th century, a ``computer'' was
still a profession, today most computers are electronic machines.  We'll use
the words ``computer'' and ``machine'' interchangably.}

\end{definition}

While it may be modern (about time) to speak of computers as machines with
multiple processors, we will restrain ourselves to single processor machines.
Programming for multiple processors raises a range of problematics irrelevant
to the core matter of these lecture notes. At times it will provide for an
interesting discussion to consider how certain aspects map over into
multiprocessor machines, but in such cases they will be mentioned explicitly.

\begin{definition}

``An action is a happening, taking place in a finite period of time and
establishing a well-defined, intended \key{net-effect}.''
\cite{dijkstra-introduction}

\end{definition}

This definition highlights two important points. Firstly, an action takes place
in a finite period of time, say $T_1$. This allows for us to speak of the
points in time $T_n$ and $T_{n+1}$, as the times of action inception and
termination, i.e. $T_{n+1}=T_n+T_1$. Secondly, an action establishes a
well-defined, intended net-effect. This highlights that we are interested in
\emph{deterministic} actions that \emph{change the state} of a computer.

To this end, it makes sense to describe the ``net-effect'' of an action as the
difference between the state of the computer at time $T_n$ and $T_{n+1}$.
However, this notion breaks down as we turn to multiprocessor machines where
the clear benefit of performing multiple actions at once has been utilized.
Therefore, apply this notion with care.

% ``action – a function or a function object that mutates the value of an
% object'' -- Alexander Stepanov, Adobe notes.

As an example of an action, consider eating lunch.
