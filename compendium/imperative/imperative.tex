An alternative programming paradigm, the \key{imperative}, is historically
closer to hardware. It emerged as the programming paradigm for computer
architectures stemming from John von Neumann's original proposal for the design
of the digital computer\cite{von-neumann}.  A key aspect of the von Neumann
architecture is that it keeps the computer rather feebleminded, and hence, it
is the programmer that has to be the clever one when specifying the program for
the computer to evoke. Inevitably, no good introduction to the imperative
programming paradigm can commence, without at least a gentle introduction to
the von Neumann architecture.
