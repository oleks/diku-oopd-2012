\section{The preliminary analysis}

The purpose of the preliminary analysis is generally stated to be to gain an
insight into the problems with the current course organization. The sort and
source of problems can vary in different ways.

\subsection{Why personal semi-structured interviews?}

In the following text we will refer to ``students'' generally as those that
have taken the course at DIKU in particular, without further distinction of the
year in which the course was taken, the grade received, etc., unless otherwise
specified.

The student focus group consists solely of students that have formerly taken
this course at DIKU.


It would also be interesting to compare the DIKU students to their peers at
other universities, and softer institutions teaching software development.

\begin{enumerate}

\item As some questions, in particular those concerning core and threshold
concepts, will take an examinatory form, it is important that the students
answer with ``whatever comes to mind'' rather than have a chance to succomb to
looking up the ``right'' answer in an external resource. It may even be
desirable to distinguish between the students that give such ``text-book
answers'' compared to those that give (clearly) self-instrumented answers.

This is also the reason why the actual interview plan should not be distributed
to the students beforehand.

\item Some students will be more proficient than others in core OO concepts or
programming concepts in general. Having this in mind, it is important for the
interviewer to have the ability to change the phrasing of the questions,
adjusting the phrasing to the particular (apparent) proficiency of the
interviewee.

\end{enumerate}



