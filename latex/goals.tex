\section{Formål}

I følgende foreslag omtales et sprog ``X'', der endnu ikke er bestemt men i
sidste ende ønskes til at være enten C\# eller Java; altså et rent
objekt-orienteret sprog, der i skrivende stund aktivt bruges ude i
erhvervslivet. X kan betragtes som både det nuværende sprog på kurset (Java),
og det sprog der vil hovedsageligt benyttes i det ny foreslaget OOPD kursus
(ikke nødvendigvis Java).

Fortsat fokus på et sprog der aktivt bruges i erhvervslivet giver os mulighed
for at uddanne studerende til at bruge professionelle værktøjer og efter
professionelle retningslinjer, men har også den ulempe at være præget af
professionel inkompetence\footnote{I form af f.eks. kringlet funktionalitet der
overlever grundet fortsatte ønsker om bagudkompatibilitet, m.v.}. Det er i
mellemtiden ikke formålet med dette forslag at behjælpe disse ulemper, men
derimod at fremlægge et pædagogisk fremgangsmåde til en introduktion af
studerende til imperativt og derefter objekt-orienteret programmering.
Udgangspunktet er altså, at et sprog som ``X'', alene ikke kan anvendes
effektivt til netop dette formål.

\section{Foreslaget i et nøddeskal}

Brug Google Go i de første par uger af kurset og brug den resterende tid på X.
Eksamensopgaven må hermed bestå af $1/3$ Go, og $2/3$ X opgaver. Formålet er
altså at bruge Go som et pædagogisk værktøj, ligesom Dr. Java\cite{dr-java} og
BlueJ\cite{bluej} har været brugt i forrige iterationer af kurset. Der
argumenteres for i dette forslag at Go er et bedre udgangspunkt, givet de
forventninger vi har til de studerende i hhv. starten og slutningen af kurset.
Der argumenteres dernæst for at der kan forekomme en blød overgang fra Go til X
ca. ved slutningen af den første 3. del af kurset.

%\subsection{Hvorfor ikke fortsætte med java?}

%Bruget af Java i OOPD kurset på datalogi har en række ulemper.

%\begin{itemize}

%\item På trods af sine stærke rødder som moren af OO-sprog med automatisk
%hukkomelsesoprydning, er Java i dag faldet bagud med udviklingen og er ved at
%blive overtaget af uforventede rivaler såsom C++11 og C\#. Dette gælder ikke
%nødvendigvis popularitet, men derimod udtrykskraft.

%\item Java har en stor bibliotek der som udgangspunkt er overfladisk for de
%studerende at lære om. Vi mener at det ville være mere nyttigt fra et
%undervisningssynspunkt for de studerende at få et indblik i hvordan man laver
%en lignende slags biblioteker selv hellere end at lære at bruge Java's
%bibliotekker.

%\item Der er en stor indlæringskurve i Java således at klasser, statiske
%metoder, stringe, og arrays må forklares \emph{inden} man forklarer for-løkker.
%Dette har man forsøgt at mitigere for både på DIKU og på andre universiteter
%ved at bruge en grafisk eller kommanodlinje fortolker.

%\item Det kan ses som et træk for at følge industrien, men formålet er tvært
%det modsatte. Vi vil gerne forholde kurset på et niveau der ikke gør de
%studerende til Java mestre, men hellere individer med indblik i impirisk
%programmering med objekt-orienterede aspekter. Formålet er altså at forholde
%kursusmaterialet på et så sproguspecifik nivau som muligt, samtidigt med at de
%studerende får lov til at øve sine kodeevner.

%\end{itemize}

\section{Hvorfor starte med Go}

Et af de mest iøjnefaldende grunde til at starte med Go er hvordan et
HelloWorld program ser ud i Go:

\begin{lstlisting}
package main

import "fmt"

func main() {
  fmt.Println("Hello World!")
}
\end{lstlisting}

Antaget at programmet for oven er gemt som \mono{HelloWorld.go}, kører man
programmet således:

\begin{verbatim}
> 6g HelloWorld.go
> 6l HelloWorld.6
> ./6.out
Hello World!
\end{verbatim}

Her er \mono{6g} en oversætter, og \mono{6l} er en sammenhæfter. De umiddelbare
fordele ved sådan en start er følgende:

\begin{itemize}

\item Det er ikke nødvendigt at introducere en fortolker for at snakke om basal
programkonstruktion da det basale program nemlig er så basal.

\item Det er ikke nødvendigt at introducere klasser, static, stringe
eller arrays for at snakke om et HelloWorld program.

\item Et koncept som en oversætter bliver iøjnefaldende.

\end{itemize}

Det er bemærkelsesværdigt at vi mener at en introduktion af en
fortolker\footnote{Grafisk eller kommandolinjebaseret, i som BlueJ eller Dr.
Java.} er ligefrem skadeligt for forståelsen af programmeringsparadigmet i X,
der nemlig er oversætterbaseret. Vi har f.eks. i 2010 oplevet studerende der
ikke vidste hvad en main funktion var ved slutningen af kurset.

Dog kan det siges at være en ulempe at man både skal oversætte og sammenhæfte
sit program før man får lov at køre det. Dette kan der dog evt. korrigeres for
ved at bruge et ekstra værktøj som f.eks.
\mono{godag}\footnote{\url{http://code.google.com/p/godag/}.}, der er en simpel
udgave af noget til Go som det velkendte værktøj \mono{ant} er for Java.

Det er dernæst oplagt at konstruere make filer (såfremt \mono{godag} ikke
benyttes). Vi mener at konstruktion af filer der eksplicit angiver rækkefølgen
og afhængigheder blandt kildekodefiler er tiden værd, da efter vores erfarring
er dette er noget man ofte støder ind i uanset om man programmerer Go, Java,
C\#, C++, el.lign.

Der er et par andre, lige så iøjnefaldende grunde som HelloWorld programmet.  I
Go benyttes der C-lignende strukturer. Objekter i Go er dermed, først og
fremmest, strukturer, altså sammensætninger af primitive typer. Selvom Go's
strukturer minder meget C's strukturer så er de dog meget nemmere at have at
gøre med end i dem i C. F.eks. er følgende en sruktur for et punkt i
to-dimensionelt rum:

\begin{lstlisting}
type Point struct {
	x, y int
}
\end{lstlisting}

Der er altså ikke eksplicitte klasser i Go, og Go er egentlig som udgangspunkt
modul-baseret (hvilket er bl.a. grunden til at main funktionen ikke skal ind i
en klasse som vi kender det fra bl.a. X). Felter (og instansmetoder) har i Go
enten privat eller offentlig synlighed, alt afhængigt af om navnet er i stort
eller småt.

Det andet grund er så at Go har pegere, men ikke pegeraritmetik, altså
referencer, blandet med en god spildopsamler\footnote{Den er ``god'' pga. noget
der nævnes om lidt.}. Her er et eksempel på brug af det ovennævnte struktur:

\begin{lstlisting}
func Origo() *Point {
	return &Point{0,0}
}
\end{lstlisting}

Så \mono{*} og \mono{\&} bruges som sædvanligt i C, men en reference kan
returneres uden at plads er blevet eksplicit allokeret til strukturen. Der er
altså en spildopsamler der kommer ind og rydder op for de kære studerende, når
``instancen'' ikke længere er i brug. Syntaksen \mono{\&Point{0,0}}, ligner
struktur-konstruktionssyntaks fra C, og i Go svarer til \mono{new Point()},
desværre er der dog ingen måde at specificere konstruktor funktionalitet.

Fordelen er således at de mest enkelte strukturer kan introduceres nemt og
hurtigt og en pæn linje kan tegnes mellem reference og værdityper, eftersom
referencer angives eksplicit. Alt dette kan gøres uden at begrave de studerende
i hukkomelsesstyring som ellers ville have været tilfældet havde det nu været
C/C++ vi beskæftigede os med.

En sidste fordel der er værd at nævne er hvor pædagogisk velplaceret syntaksen
er for at deklarere instansmetoder. Først og fremmest er der naturligvis
mulighed for at smide en reference som en argument til en funktion.
Objekt-orienteret programming, især i form af punktum-syntaks kan hermed introduceres
som et værktøj til at spare en eksplicit angivelse af et reference argument.

En instansmetode defineres således:

\begin{lstlisting}
func (*Point point) AddX(value int) int {
	point.x += value
}
\end{lstlisting}

Og kaldes således:

\begin{lstlisting}
p = Orego()
p.AddX(5)
\end{lstlisting}

Der er hermed nemt at tegne en linje mellem statiske og instansmetoder. Dernæst
har instansmetoder, ligesom strukturens felter kan have enten privat eller
offentlig synlighed alt afhængigt om navnet starter med stort eller småt.

\section{Hvorfor ikke fortsætte med Go}

Ud over det basale empiriske programmering brydder Go mange af de aspekter der
ellers har været undervist, og bør undervises i et kursus med navnet
``objekt-orienteret programmering og design''. Specifikt har Go hverken
indbygget understøttelse af type-hierarkier og parametrisk polimorfi. Det
første er noget Go's udviklere står imod og det andet har ikke fundet sin vej
ind i sproget endnu\cite{go-faq}. Det ville uden tvivl være nødvendigt at
overveje at skifte helt til Go når, og hvis sproget får parametrisk polimorfi.

Derudover er der umiddelbart følgende mangler:

\begin{itemize}

\item Ingen syntaktisk sukker for struktur konstruktører.

\item Ingen metode overlasning.

\item Ingen dynamic dispatch.

\item Interfaces fungerer på en grundlæggende anderledes måde.

\end{itemize}

\section{Overgang til X} 

Følgende koncepter kan som anvist for oven introduceres vha. Go på en
evolutionær\footnote{I som at alt hvad der foregår i en kodefil kan forklares
med det samme, dvs. at der ikke er nogen overflødelige tegn og ord involveret.}
måde:

\begin{enumerate}

\item main funktionen

\item oversætter

\item modul afhængighedsstyring

\item statiske metoder

\item strukturer / objekter

\item primitive datatyper inkl. stringe

\item privat / offentlig synlighed af instansvariable og metoder

\item arrays (under spørgsmål\footnote{Der er nogle syntaktiske vanskeligheder
omkring arrays i Go, der er endnu ikke fundet en god intuitiv afgrænsning så de
``minder om'' arrays som vi ser i X.})

\item if, for løkker

\item konstanter

\item simple datastrukturer

\end{enumerate}

Ved slutningen af denne periode med Go, kan de studerende forventes at kunne
skrive f.eks. en hægtet liste samt dynamisk array strukturer uden besvær.

Med disse værktøjer i værktøjsskuffen kan Java introduceres på praktisk talt
3-5 slides. I følgende sektioner, lad kodesekvenserne være placeret i to
kolonner for hhv. Go og Java i hver slide.

\subsection{En klasse i Go vs. en klasse i Java}

\begin{lstlisting}
type Point struct {
  x, y int
  Z int
}

func (point *Point) Move(x, y int) {
  point.x += x
  point.y += y
  point.Z = point.x * point.y
}
\end{lstlisting}

\begin{lstlisting}[language=Java]
public class Point
{
  private int x, y;
  public int Z;

  public Move(int x, int y)
  {
    this.x += x;
    this.y += y;
    Z = this.x * this.y;
  } 
}
\end{lstlisting}

\begin{itemize}

\item Klassespecifikation vs. metodespecifikation.

\item \mono{point} vs. \mono{this}. 

\item \mono{Z} kan tilgås, så \mono{this} kan være gratis.

\end{itemize}

\newpage

\subsection{En main funktion i Go vs. en main funktion i Java}

\begin{lstlisting}
package main

func main()
{
  // ...
}
\end{lstlisting}

\begin{lstlisting}[language=Java]
public class Main
{
  public static void main(String[] arguments)
  {
    // ...
  }
}
\end{lstlisting}

\begin{itemize}

\item \mono{main} skal være i en klasse.

\item \mono{main} skal have en denne signatur med argument array'et angivet.

\end{itemize}

\newpage

\subsection{Initializering i Go vs. Java}

\begin{lstlisting}
type Point struct {
  x, y int
}

func main()
{
  point = new(Point) // sets all values to 0.
}
\end{lstlisting}

\begin{lstlisting}[language=Java]
public class Point
{
  private int x, y;
}

public class Main
{
  public static void main(String[] arguments)
  {
    Point point = new Point(); // sets all values to 0.
  }
}
\end{lstlisting}

\begin{itemize}

\item automatisk tilføjelse af en konstruktør. 

\end{itemize}

\newpage

\subsection{Initializering i Go vs. Java \# 2}

\begin{lstlisting}
type Point struct {
  x, y int
}

func main()
{
  point = &Point{3,5}
}
\end{lstlisting}

\begin{lstlisting}[language=Java]
public class Point
{
  private int x, y;

  public Point(int x, int y) // "constructor"
  {
    this.x = x;
    this.y = y;
  }
}

public class Main
{
  public static void main(String[] arguments)
  {
    Point point = new Point(3,5);
  }
}
\end{lstlisting}

\begin{itemize}

\item mulighed for at specificere konstruktører.

\item udelukkelse af tomme konstruktører ved tilstedeværelse af parametriserede
konstruktører.

\item evt. at en tom konstruktør (uden kode) vil igen tillade os at bygge
objektet således at alle værdier sættes til 0. Der kan evt. beskrives
fremgangsmåden for en klasses initializering fuldt ud, hvorved der evt. kan
nævnes hvorfor man ikke skal intializere sine variabler i toppen af klassen.

\end{itemize}

